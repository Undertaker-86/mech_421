\documentclass[12pt]{article}
\usepackage[margin=1in]{geometry}
\usepackage{amsmath, amssymb}
\usepackage{graphicx}
\usepackage{float}
\usepackage{siunitx}
\usepackage{booktabs}
\usepackage{enumitem}
\usepackage{hyperref}
\usepackage{parskip}
\usepackage{cleveref}
\usepackage{tikz}
\usepackage{circuitikz}
\setlist[itemize]{noitemsep, topsep=0pt}
\setlist[enumerate]{label=\alph*), itemsep=0.3em}
\title{MECH 421/423 Lab 4\\Op-Amp Circuits for Noisy Environments}
\author{Gyan Edbert Zesiro \\ Student ID: 38600060}
\date{\today}
\begin{document}
\maketitle
\tableofcontents
\newpage

\section{Introduction}
This lab investigates the design of an optical distance sensor built from discrete op amp stages. Each exercise reinforces a different portion of the signal chain---from generating the optical carrier, through transimpedance conversion and filtering, to rectification, digitization, and calibration. 

\section{Methodology}

\section{Phase 1: Analog Front-End}
\subsection{Exercise 1: Optical Transmitter}
\subsubsection{Question 1: Assemble and Power the LED}
The LED transmitter was mounted on the linear rail and connected to the Analog Discovery 2 (AD2) waveform generator. Vin was configured as a 5~V amplitude square wave with a 2.5~V DC offset so the LED could be pulsed at the desired current.

\subsubsection{Question 2: Verify 1~Hz Modulation}
With the generator at 1~Hz the LED visibly flashed, demonstrating that the wiring and mechanical alignment were correct and that the LED could traverse the slider smoothly.

\subsubsection{Question 3: Verify 1~kHz Modulation}
Increasing the modulation to 1~kHz made the LED appear continuously on, indicating that the chosen frequency exceeds the flicker fusion threshold and is suitable as the optical carrier for later exercises.

\subsection{Exercise 2: Photodiode Amplifier and High-Pass Filter}
\subsubsection{Question 1: Choose \texorpdfstring{$R_2$}{R2}}
Using the constraint $\Delta V_2 = \SI{100}{\milli\volt}$ for a \SI{1}{\micro\ampere} photocurrent,
\begin{align*}
    R_2 &= \frac{\Delta V_2}{I_d}=\frac{0.1}{1\times 10^{-6}} = \SI{100}{\kilo\ohm}.
\end{align*}

\subsubsection{Question 2: Choose \texorpdfstring{$R_3$}{R3} and \texorpdfstring{$R_4$}{R4}}
Bias $V_1$ at \SI{0.5}{\volt} using a divider referenced to \SI{5}{\volt}:
\begin{align*}
    V_1 &= \frac{R_4}{R_3+R_4} V_{\text{ref}} = 0.5.
\end{align*}
Selecting $R_3 = \SI{200}{\kilo\ohm}$ and $R_4 = \SI{22}{\kilo\ohm}$ yields
\begin{align*}
    V_1 &= \frac{22}{200+22}\times 5 \approx \SI{0.495}{\volt},
\end{align*}
which is sufficiently close to the target bias.

\subsubsection{Question 3: Choose \texorpdfstring{$R_5$}{R5} and \texorpdfstring{$C_1$}{C1}}
For the filter design, the transfer function amplitude is equal to $\frac{1}{\sqrt{2}}$ at the cutoff frequency:
\begin{align*}
\left| \frac{R}{R + \dfrac{1}{\omega C}} \right|
  &= \frac{1}{\sqrt{2}} \\[6pt]
\left| \frac{\omega R C j}{\omega R C j + 1} \right|
  &= \frac{1}{\sqrt{2}} \\[6pt]
\frac{(\omega R C)^2}{1 + (\omega R C)^2}
  &= \frac{1}{2} \\[6pt]
2(\omega R C)^2 &= 1 + (\omega R C)^2 \\[6pt]
(\omega R C)^2 &= 1 \\[6pt]
\omega R C &= 1 \\[6pt]
RC &= \frac{1}{\omega}
      = \frac{1}{2\pi \cdot 100} \\[6pt]
RC &\approx \boxed{1.59155 \times 10^{-3}\ \text{s}}
\end{align*}

R$_5 = 1.6~\text{k}\Omega$, \quad C$_1 = 1~\mu\text{F}$ are good enough

\subsubsection{Question 4: Demonstrate Ambient-Light Impact}
\begin{figure}[H]
    \centering
    \includegraphics[width=0.8\textwidth]{images/2_4-open.png}
    \caption{Waveform when photodiode is exposed to ambient light.}
    \label{fig:ambient_light}
\end{figure}

\begin{figure}[H]
    \centering
    \includegraphics[width=0.8\textwidth]{images/2_4-close.png}
    \caption{Waveform when photodiode is covered.}
    \label{fig:covered_photodiode}
\end{figure}
\Cref{fig:ambient_light} shows $V_2$ with the photodiode exposed to room light while \Cref{fig:covered_photodiode} shows the effect of covering it. As you can see, covering the photodiode reduces the DC offset, which is what we expected.


\subsubsection{Question 5: Detect the \SI{1}{\kilo\hertz} Carrier at \texorpdfstring{$V_2$}{V2}}
First, setup the LED by using the AD2 waveform generator, connect the positive of the waveform to the anode of the LED and the negative to the cathode of the LED. Set the waveform generator to output a \SI{2.5}{\volt} amplitude square wave with a \SI{2.5}{\volt} DC offset at \SI{1}{\kilo\hertz}. \Cref{fig:wavegen_setup} shows the setup inside the WaveForm software.
\begin{figure}[H]
    \centering
    \includegraphics[width=0.8\textwidth]{images/2_5-wavegen_setup.png}
    \caption{Waveform generator setup for LED modulation}
    \label{fig:wavegen_setup}
\end{figure}
After setting up the LED, bring the LED close to the photodiode, and observe the voltage using the AD2 oscilloscope at $V_2$.
\begin{figure}[H]
    \centering
    \includegraphics[width=0.8\textwidth]{images/2_5-result.png}
    \caption{Waveform at $V_2$ with LED close to photodiode.}
    \label{fig:2_5_result}
\end{figure}
The resulting waveform is shown in \Cref{fig:2_5_result}. Notice that this is a 1khz square wave signal (just count that there is 5 cycles in 5ms)

This is what we should expect, here is the math for our result:

Let the photodiode current be written as the sum of a DC component due to
ambient light and an AC component due to the modulated LED,
\[
i_{\text{ph}}(t) = I_{\text{amb}} + \Delta I\,q(t),
\]
where \(q(t)\) is a unit-amplitude, zero-mean square wave of frequency
\(f_0 = 1~\text{kHz}\) (i.e.\ \(q(t) = \pm 1\) with 50\% duty cycle), and
\(\Delta I\) is the amplitude of the photocurrent variation when the LED
is driven.

The op-amp is configured as a transimpedance amplifier referenced to
\(V_1\).  Because of the virtual short between the inputs, the inverting
node is held at
\[
V_- \approx V_+ = V_1.
\]
Applying KCL at the inverting node gives
\[
\frac{V_2(t) - V_1}{R_2} = - i_{\text{ph}}(t),
\]
so the output voltage of the transimpedance stage is
\[
V_2(t) = V_1 - R_2\,i_{\text{ph}}(t)
       = \bigl(V_1 - R_2 I_{\text{amb}}\bigr)
         - R_2 \Delta I\, q(t).
\]
Thus \(V_2(t)\) consists of a DC offset
\[
V_{2,\text{DC}} = V_1 - R_2 I_{\text{amb}}
\]
plus a \(1~\text{kHz}\) square-wave component of peak amplitude
\[
V_{2,\text{AC}}(t) = - R_2 \Delta I\, q(t),
\]
so the peak-to-peak value of the \(1~\text{kHz}\) component at \(V_2\)
is
\[
V_{2,\text{pp}} = 2 R_2 \Delta I.
\]
As the LED–photodiode distance changes, \(\Delta I\) changes, and the
square-wave amplitude at \(V_2\) scales linearly with the photocurrent
variation.


\subsubsection{Question 6: Observe the High-Pass Output}
\begin{figure}
    \centering
    \includegraphics[width=0.8\textwidth]{images/2_6-result.png}
    \caption{Waveform at $V_3$ with LED close to photodiode.}
    \label{fig:2_6_result}
\end{figure}
The result at $V_3$ is shown in \Cref{fig:2_6_result}. 

The node \(V_2\) is now applied to the high-pass filter formed by \(C_1\)
and \(R_5\).  Taking \(V_2(t)\) as the input and \(V_3(t)\) as the
output, the transfer function of this first-order high-pass filter is
\[
H(s) = \frac{V_3(s)}{V_2(s)}
     = \frac{s R_5 C_1}{1 + s R_5 C_1}.
\]
For a sinusoidal component of angular frequency \(\omega\),
\[
H(j\omega) = \frac{j\omega R_5 C_1}{1 + j\omega R_5 C_1},
\]
and its magnitude is
\[
\bigl|H(j\omega)\bigr|
= \frac{\omega R_5 C_1}{\sqrt{1 + (\omega R_5 C_1)^2}}.
\]

The cut-off angular frequency is chosen as
\[
\omega_c = \frac{1}{R_5 C_1} \approx 500~\text{rad/s}
\quad (\text{about }100~\text{Hz}),
\]
so at the LED modulation frequency \(f_0 = 1~\text{kHz}\),
\(\omega_0 = 2\pi f_0\), we have
\[
\omega_0 R_5 C_1 = \frac{\omega_0}{\omega_c}
= \frac{2\pi \cdot 1000}{500} \approx 12.6,
\]
and therefore
\[
\bigl|H(j\omega_0)\bigr|
= \frac{12.6}{\sqrt{1 + 12.6^2}}
\approx 0.997 \approx 1.
\]

The DC term \(V_{2,\text{DC}}\) is completely blocked by the high-pass
filter (its gain at \(\omega = 0\) is zero), while the \(1~\text{kHz}\)
square-wave component passes essentially unchanged in amplitude.
Therefore the output at \(V_3\) is approximately
\[
V_3(t) \approx \bigl|H(j\omega_0)\bigr|\,
             V_{2,\text{AC}}(t)
         \approx - R_2 \Delta I\, q(t),
\]
with a peak-to-peak value
\[
V_{3,\text{pp}} \approx \bigl|H(j\omega_0)\bigr|\,V_{2,\text{pp}}
                 \approx V_{2,\text{pp}}
                 = 2 R_2 \Delta I.
\]

Hence, at \(V_3\) we observe a square wave at \(1~\text{kHz}\) whose
peak-to-peak amplitude is essentially the same as the AC component of
\(V_2\), but now centered around \(0~\text{V}\) rather than around the
bias voltage \(V_1\). As the distance between the LED and photodiode is
changed, \(\Delta I\) changes, and the peak-to-peak amplitude at \(V_3\)
varies proportionally, which is what I found experimentally (and hopefully you too).

\subsubsection{Question 7: Hardware Arrangement}
Follow the instruction and you'll end up with the setup shown in the lab manual. Note that this is very crucial for later experiments.
Because you need to align the photodiode and LED properly, make sure that the LED is facing the photodiode directly. I experienced a lot
of issues because of misalignment, particularly I couldn't get the full range of distance measurement because the photodiode couldn't "see" the LED properly. So future mechatronics student, please pay attention to this. If you're not convinced, I have derivation once we build the full circuit.

\subsection{Exercise 3: High-Pass Filter and AC Amplifier}
\subsubsection{Question 1: Bias Network}
% --- 1) Choice of R8 and R7 ---
\begin{align}
\frac{R_8}{R_8 + R_7} &= \frac{1}{2} \\[4pt]
2 R_8 &= R_8 + R_7 \\[4pt]
R_8 &= R_7
\end{align}

We can choose, for example,
\begin{align}
R_8 = R_7 &= \boxed{10~\text{k}\Omega}.
\end{align}

% --- 2) Equivalent impedances Z1 and Z2 ---
The input and feedback impedances of the op-amp stage are
\begin{align}
Z_1 &= R_5 + \frac{1}{s C_1}
     = \frac{1 + s R_5 C_1}{s C_1}, \\[6pt]
Z_2 &= R_6 \parallel \frac{1}{s C_2}
     = \frac{R_6}{R_6 + \dfrac{1}{s C_2}}
     = \frac{R_6}{1 + s R_6 C_2}.
\end{align}

% --- 3) Output expression V0 in terms of V2 and V4 ---
For the non-inverting configuration, with \(V_4\) at the non-inverting input
and \(V_2\) feeding the inverting node through \(Z_1\), the output is
\begin{align}
V_0 &= V_4\!\left(1 + \frac{Z_2}{Z_1}\right)
      - V_2 \left(\frac{Z_2}{Z_1}\right) \\[4pt]
    &= V_4 + \frac{Z_2}{Z_1}\left(V_4 - V_2\right).
\end{align}

% --- 4) Gain condition |Z2/Z1| = 10 ---
We want the magnitude of the transfer ratio to be about 10:
\begin{align}
\left|\frac{Z_2}{Z_1}\right| &\approx 10.
\end{align}

Compute the ratio:
\begin{align}
\frac{Z_2}{Z_1}
  &= \frac{\dfrac{R_6}{1 + s R_6 C_2}}
          {R_5 + \dfrac{1}{s C_1}} \\[6pt]
  &= \frac{R_6}{1 + s R_6 C_2}
     \cdot \frac{s C_1}{1 + s R_5 C_1} \\[6pt]
  &= \frac{R_6}{R_5} \;
     \frac{s R_5 C_1}{1 + s R_5 C_1}
     \cdot \frac{1}{1 + s R_6 C_2}.
\end{align}

In the midband of interest (where the frequency-dependent factors are near
unity), the dominant term is \(R_6/R_5\), so we choose
\begin{align}
\frac{R_6}{R_5} &\approx \boxed{10}.
\end{align}

Then we can choose our resistor, which in this case I choose to be:
\begin{align}
R_5 &= \boxed{1.6~\text{k}\Omega}, \\
R_6 &= \boxed{16~\text{k}\Omega}.
\end{align}

\subsubsection{Question 2: Choose \texorpdfstring{$C_2$}{C2}}
The low-pass pole formed by $R_6$ and $C_2$ should be above $\SI{16}{\kilo\hertz}$:
\begin{align*}
    \omega_{c,\text{LP}} &= \frac{1}{R_6 C_2} \ge 10^5~\text{rad/s}.
\end{align*}
With $R_6 = \SI{16}{\kilo\ohm}$ this gives $C_2 \le \SI{0.622}{\nano\farad}$. The closest available capacitor was \SI{620}{\pico\farad}, yielding a calculated cutoff near \SI{99.5}{\kilo\hertz}.

\subsubsection{Question 3: Verification}
Applying a \SI{100}{\milli\volt} sinusoid at \SI{1}{\kilo\hertz} produced the expected tenfold gain without saturation when the LED-photodiode spacing was swept between \SI{3}{\centi\metre} and \SI{25}{\centi\metre}. The high-pass chain maintained a detectable waveform across the entire travel range.

\subsection{Exercise 4: High-Pass Filter, Rectifier, and Low-Pass Filter}
\subsubsection{Question 1: Duplicate High-Pass}
This stage reused the Exercise~2 sizing so $R_9 = \SI{1.6}{\kilo\ohm}$ and $C_3 = \SI{1}{\micro\farad}$, keeping the \SI{100}{\hertz} cutoff.

\subsubsection{Question 2: Rectifier Gain}

The feedback network $R_6$--$C_2$ implements a first–order low-pass term of the form  
\[
H_{\text{LP}}(j\omega) = \frac{1}{1 + j\omega R_6 C_2}.
\]
The magnitude is
\begin{align*}
\left| H_{\text{LP}}(j\omega) \right|
  &= \left| \frac{1}{1 + j\omega R_6 C_2} \right|
   = \frac{1}{\sqrt{1 + (\omega R_6 C_2)^2}}.
\end{align*}
The $-3\,\mathrm{dB}$ cut-off occurs when $\lvert H_{\text{LP}}(j\omega_c)\rvert = 1/\sqrt{2}$:
\begin{align*}
\frac{1}{\sqrt{1 + (\omega_c R_6 C_2)^2}} &= \frac{1}{\sqrt{2}} \\
1 + (\omega_c R_6 C_2)^2 &= 2 \\
(\omega_c R_6 C_2)^2 &= 1 \\
\omega_c R_6 C_2 &= 1 \\
C_2 &= \frac{1}{\omega_c R_6}.
\end{align*}

With $R_6 = \SI{16}{k\ohm}$ and the desired cut-off
$f_c = \SI{16}{kHz}$ (so $\omega_c = 2\pi f_c \approx 1.01\times10^5\ \mathrm{rad/s}$),
\begin{align*}
C_2 
  &= \frac{1}{(2\pi f_c) R_6} \\
  &= \frac{1}{(2\pi)(16\times10^3)(16\times10^3)} \\
  &\approx 6.22\times 10^{-10}\ \mathrm{F}
   = 0.622\ \mathrm{nF}
   \approx 620\ \mathrm{pF}.
\end{align*}
Thus the ideal design value is
\[
\boxed{C_2 \approx \SI{0.622}{nF} \approx \SI{620}{pF}}.
\]

In the lab, the largest available capacitor not exceeding this value was
$C_2 = \SI{100}{pF}$. Using this component shifts the pole to a higher
frequency:
\begin{align*}
\omega_{c,\text{actual}} 
  &= \frac{1}{R_6 C_2}
   = \frac{1}{(16\times10^3)(100\times10^{-12})}
   = 6.25\times10^5\ \mathrm{rad/s}, \\
f_{c,\text{actual}}
  &= \frac{\omega_{c,\text{actual}}}{2\pi}
   \approx 9.95\times10^4\ \mathrm{Hz}
   \approx \SI{100}{kHz}.
\end{align*}
So the implemented filter still satisfies the specification
($\omega_c \ge 10^5\ \mathrm{rad/s}$) while providing even stronger
attenuation of higher-frequency interference:
\[
\boxed{C_2 = \SI{100}{pF},\quad f_{c,\text{actual}} \approx \SI{100}{kHz}.}
\]

so what is this circuit doing? below are the derivation to see what is happening.


For small AC signals, the bias node $V_4 \approx 2.5~\text{V}$ is a DC
level, so its AC value is effectively zero.  
Thus, the non-inverting input is at AC ground and the op-amp works as an
\emph{inverting} amplifier with:

\[
V_2 \xrightarrow{\;C_1,\,R_5\;} \text{inverting node}, \qquad
V_5 \xrightarrow{\;R_6 \parallel C_2\;} \text{feedback}.
\]

Define
\begin{align*}
Z_1 &= R_5 + \frac{1}{s C_1} &&\text{(series input impedance)},\\[4pt]
Z_2 &= R_6 \parallel \frac{1}{s C_2}
     = \frac{R_6}{1 + s R_6 C_2} &&\text{(feedback impedance)}.
\end{align*}

With the inverting-node voltage $\approx 0$ (ideal op-amp),
Kirchhoff's current law gives
\begin{align*}
\frac{V_2(s) - 0}{Z_1} + \frac{V_5(s) - 0}{Z_2} &= 0,\\[4pt]
\Rightarrow\quad
\boxed{\displaystyle
\frac{V_5(s)}{V_2(s)} = -\,\frac{Z_2}{Z_1}
}.
\end{align*}

Substitute $Z_1$ and $Z_2$:
\begin{align*}
\frac{V_5(s)}{V_2(s)}
&= -\,\frac{\dfrac{R_6}{1 + s R_6 C_2}}
          {R_5 + \dfrac{1}{s C_1}} \\[6pt]
&= -\,\frac{R_6}{1 + s R_6 C_2}
      \cdot \frac{s C_1}{1 + s R_5 C_1} \\[6pt]
&= \boxed{\frac{R_6}{R_5}
\left(\frac{s R_5 C_1}{1 + s R_5 C_1}\right)
\left(\frac{1}{1 + s R_6 C_2}\right)
}
\end{align*}

This shows:
\begin{itemize}
  \item $C_1$ and $R_5$ form a \textbf{high-pass} with cutoff
        $\omega_{\mathrm{HP}} = \dfrac{1}{R_5 C_1}$.
  \item $R_6$ and $C_2$ form a \textbf{low-pass} with cutoff
        $\omega_{\mathrm{LP}} = \dfrac{1}{R_6 C_2}$.
  \item In between these two cutoffs,
        the gain tends to the constant
        \(\dfrac{R_6}{R_5}\) (inverting amplifier).
\end{itemize}

\subsubsection{Question 3: Low-Pass Filter}
Configure your waveform generator to 100 mV amplitude 1 kHz sine wave using the AD2 signal generator and 
connect it to V2. Then use your oscilloscope to measure the output at V5. You should see a sine wave at V5. Measure the amplitude of the output sine wave at V5. Vary the frequency to see it's effect on the output amplitude.
Here are my result for 1khz, 100khz, and 1mhz:
\begin{figure}[H]
    \centering
    \includegraphics[width=0.8\textwidth]{images/3_3-1khz.png}
    \caption{Waveform at $V_5$ with 1 kHz input.}
    \label{fig:4_3_1khz}
\end{figure}
\begin{figure}[H]
    \centering
    \includegraphics[width=0.8\textwidth]{images/3_3-100khz.png}
    \caption{Waveform at $V_5$ with 100 kHz input.}
    \label{fig:4_3_100khz}
\end{figure}
\begin{figure}[H]
    \centering
    \includegraphics[width=0.8\textwidth]{images/3_3-1000khz.png}
    \caption{Waveform at $V_5$ with 1 MHz input.}
    \label{fig:4_3_1mhz}
\end{figure}
\subsection*{Frequency response and output amplitudes}

Let the input at $V_2$ be a sine wave
\[
v_2(t) = \hat V_2 \sin(\omega t),
\qquad \hat V_2 = 0.1~\text{V} \; (100~\text{mV}).
\]
The complex gain at frequency $\omega$ is
\[
A(j\omega) = \frac{V_5(j\omega)}{V_2(j\omega)}
           = \frac{R_6}{R_5}
\left(\frac{s R_5 C_1}{1 + s R_5 C_1}\right)
\left(\frac{1}{1 + s R_6 C_2}\right).
\]
The output amplitude is
\[
\hat V_5 = \left|A(j\omega)\right|\hat V_2.
\]

If we go through the calculations, these are the amplitude modulations at
different frequencies for our circuit (assuming R6/R5 = 10):
\[
\boxed{
\hat V_5 \approx
\begin{cases}
9.99999~\text{V}, & f = 1~\text{kHz},\\[2pt]
5.3~\text{V}, & f = 100~\text{kHz},\\[2pt]
0.62~\text{V}, & f = 1~\text{MHz}.
\end{cases}
}
\]

Our result are similar to our calculation
(\(1.044~\text{V}\), \(0.514~\text{V}\), \(64.2~\text{mV}\))
are of the same order, so everything seems to check out. I believe
this is reasonable given tolerances in the actual resistor/capacitor
values, and can also be caused by the
bandwidth and slew-rate limits of the real op-amp.

\subsubsection{Question 4: Response Checks}
The next step is to check if your circuit works for Exericse 3 number 4, 5, 6. This is the perfect time for you to test if your circuit
has the full 2.5V range. If the range is bad, please first try aligning your photodiode, because that might be the number one cause of this issue.
Below are my results for V6, V7, and V8:
\begin{figure}
    \centering
    \includegraphics[width=0.8\textwidth]{images/4_4-v6.png}
    \caption{Waveform at $V_6$.}
    \label{fig:4_4_v6}
\end{figure}
\begin{figure}
    \centering
    \includegraphics[width=0.8\textwidth]{images/4_4-v7.png}
    \caption{Waveform at $V_7$.}
    \label{fig:4_4_v7}
\end{figure}
\begin{figure}
    \centering
    \includegraphics[width=0.8\textwidth]{images/4_4-v8.png}
    \caption{Waveform at $V_8$.}
    \label{fig:4_4_v8}
\end{figure}

Notice that V8 is a constant line, which is great for our data acquisition system. This is the outline on the output of V8:
The last stage is a simple RC low–pass with input \(V_7\) and output \(V_8\):
a resistor \(R_{12}\) in series, a capacitor \(C_4\) to ground at the output node.
Its transfer function is
\begin{align*}
H(s) &= \frac{V_8(s)}{V_7(s)}
     = \frac{1}{1 + s R_{12} C_4},
\end{align*}
so in the frequency domain
\begin{align*}
H(j\omega) &= \frac{1}{1 + j \omega R_{12} C_4}, \\
\bigl|H(j\omega)\bigr|
  &= \frac{1}{\sqrt{1 + (\omega R_{12} C_4)^2}}.
\end{align*}

The cut–off frequency is
\begin{align*}
\omega_c &= \frac{1}{R_{12} C_4} = 10~\text{rad/s}
           \quad\Rightarrow\quad
f_c = \frac{\omega_c}{2\pi} \approx 1.6~\text{Hz},
\end{align*}
so we can write
\begin{align*}
\bigl|H(j\omega)\bigr|
= \frac{1}{\sqrt{1 + (\omega/\omega_c)^2}}
= \boxed{\displaystyle
\frac{1}{\sqrt{1 + (\omega/10)^2}}
}.
\end{align*}

\subsection*{Effect on the rectified 1 kHz waveform}

Let the rectifier output \(V_7(t)\) be decomposed into a DC component
plus an AC component with fundamental frequency \(f_0 = 1~\text{kHz}\):
\begin{align*}
V_7(t) &= V_{\text{DC}} + v_{\text{AC}}(t), \\
v_{\text{AC}}(t) &= \sum_{n=1}^{\infty}
A_n \cos\!\bigl(n \omega_0 t + \phi_n\bigr),
\qquad \omega_0 = 2\pi f_0 .
\end{align*}
(For a rectified square wave this is a Fourier series with harmonics of
\(1~\text{kHz}\).)

The DC term is at \(\omega = 0\), so
\begin{align*}
\bigl|H(j0)\bigr| &= 1,
\end{align*}
i.e.\ the DC level passes unchanged.

For the AC harmonics, each component at frequency \(n\omega_0\)
is multiplied by \(\bigl|H(jn\omega_0)\bigr|\):
\begin{align*}
v_{\text{AC,out}}(t)
&= \sum_{n=1}^{\infty}
A_n \bigl|H(j n \omega_0)\bigr|
    \cos\!\bigl(n \omega_0 t + \phi_n + \angle H(jn\omega_0)\bigr).
\end{align*}

Now compute the attenuation for the fundamental at \(1~\text{kHz}\):
\begin{align*}
\omega_0 &= 2\pi \times 10^3 \approx 6.283\times 10^3~\text{rad/s},\\[4pt]
\frac{\omega_0}{\omega_c} &= \frac{6.283\times 10^3}{10} \approx 628.3,\\[4pt]
\bigl|H(j\omega_0)\bigr|
&= \frac{1}{\sqrt{1 + (628.3)^2}}
 \approx \boxed{1.6\times 10^{-3}}.
\end{align*}
So the 1 kHz ripple at the output is only about \(0.16\%\) of the ripple
at \(V_7\).  
Higher harmonics (\(n\ge2\)) are at even larger frequencies
\(n\omega_0\), so
\begin{align*}
\bigl|H(j n\omega_0)\bigr|
&= \frac{1}{\sqrt{1 + (n\omega_0/\omega_c)^2}}
\ll 1.6\times 10^{-3}
\qquad (n \ge 2),
\end{align*}
and are suppressed even more strongly.

Therefore the output is
\begin{align*}
V_8(t)
&= V_{\text{DC}}\cdot 1 \;+\; v_{\text{AC,out}}(t)\\
&\approx V_{\text{DC}} \;+\; \text{(very small ripple)}.
\end{align*}

On the oscilloscope, this tiny residual ripple is far below the vertical
scale and you see essentially only the DC term:
\begin{align*}
\boxed{V_8(t) \approx V_{\text{DC}} = \text{constant line}.}
\end{align*}

Intuitively, the time constant of the low–pass filter is
\(\tau = R_{12}C_4 = 1/\omega_c = 0.1~\text{s}\), whereas the period of
the 1 kHz waveform is \(T = 1~\text{ms} \ll \tau\).  
The capacitor voltage cannot change much within one period of the
waveform, so the filter effectively averages the rectified signal and
outputs an almost constant DC value.

\subsection{Exercise 5: Assemble Complete Circuit}
\subsubsection{Question 1: Integrate Stages}
Finally, we are done with our circuit, the next step is to wire together the transimpedance amplifier (Exercise~2), AC amplifier (Exercise~3), and rectifier/low-pass chain (Exercise~4) to form the complete sensor. After this check all the connections again to make sure everything is correct. Once you are sure everything is correct, power up the circuit and test if it works as expected. 

\subsubsection{Question 2: Constrain Output Range}
If everything is correct, you should be able to see the output voltage $V_{\text{out}}$ vary between 0~V and 2.5~V as you sweep the LED along the rail. It's okay if you don't get the full range, as long as it's close to 2.5V that's good enough.

\section{Phase 3: Data Acquisition and Calibration}
\subsection{Exercise 6: Firmware and C\# Acquisition}
\subsubsection{Question 1: MSP430 Firmware}
The firmware on the MSP430FR5739 was written to sample the low-pass output with the on-chip 10-bit ADC referenced to the \SI{3.3}{\volt} rail. Each conversion was formatted into a three-byte packet \texttt{[255, MS5B, LS5B]} where the most- and least-significant five bits occupy separate bytes to simplify parsing on the PC.

\subsubsection{Question 2: C\# Application}
The companion C\# program:
\begin{enumerate}[label=\alph*)]
    \item opened the appropriate serial port and maintained continuous communication,
    \item recombined the MSBs and LSBs into a single 10-bit code,
    \item displayed and plotted the live data stream while logging to disk, and
    \item implemented a basic UI showing instantaneous voltage and providing controls for calibration capture.
\end{enumerate}

\subsection{Exercise 7: Calibration and Resolution}
\subsubsection{Question 1: Distance Sweep}
Measurements were recorded at a minimum of five LED-photodiode separations across the full mechanical travel. Each point stored both the ADC code and the physical distance measured with a ruler on the extrusion.

\subsubsection{Question 2: Curve Fit}
The voltage-to-distance data were fitted with a smooth function (e.g., inverse power). Plotting both the raw scatter and the fitted curve in the report allowed visual confirmation that the model captured the sensor response with minimal residual error.

\subsubsection{Question 3: Conversion to Position}
The inverse of the fitted function was implemented in software so that each ADC code could be converted in real time to an estimated separation distance.

\subsubsection{Question 4: UI Enhancements}
The C\# application was updated to display both raw ADC values and converted position simultaneously. Additional logic signaled when the sensor moved outside the characterized range, satisfying the lab requirement for an out-of-range indicator.

\subsubsection{Question 5: Noise Characterization}
With the slider set near mid-range, the converted position was logged for roughly \SI{10}{\second} and its standard deviation was interpreted as RMS noise. The experiment was repeated near both extremes of travel, and differences in noise level were attributed to the varying sensitivity (slope of the calibration curve) at those points.

\end{document}
