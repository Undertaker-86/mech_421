\documentclass[11pt]{article}
\usepackage[margin=1in]{geometry}
\usepackage{enumitem}
\usepackage{microtype}
\usepackage{graphicx}
\usepackage{listings}

\lstset{basicstyle=\ttfamily\small, frame=single}

\title{Concept Proposal: Mini Morphobot-Inspired Mobility Platform}
\author{Gyan Edbert Zesiro (ID:38600060) \and Ryan Edric Nashota (ID:33508219)}
\date{\today}

\begin{document}
\maketitle

\section{Concept Overview}
\subsection{Goal and Vision}
We will build a laptop-scale ``Mini Morphobot'' that emulates Caltech's Multi-Modal Mobility Morphobot (M4) but narrows the scope to two high-impact modes: agile wheeled driving indoors and short-hop quadrotor flight outdoors.\footnote{E.~Sabree et~al., ``Multi-Modal Mobility Morphobot (M4) with appendage repurposing for locomotion plasticity enhancement,'' \textit{Nature Communications}, 2023.} Proving that one set of rotor-wheel appendages can deliver both functions within three months would validate a deployable scout that can roll through doorways, collect samples, and then lift off to bypass rubble piles or elevator shafts without swapping hardware.

\subsection{Form Factor and Operating Concept}
The platform uses four identical convertible rotor-wheel pods arranged in an ``X'' planform on a 28~cm $\times$ 28~cm carbon chassis. Each pod nests a 180~mm ducted rotor with a tire bonded to the rim. In Ground Mode, the ducts lock at $0^\circ$ and the hub motors drive the tires for smooth wheeled motion with active suspension. In Flight Mode, the pods articulate to a $45^\circ$ dihedral, propellers spin up, and the same hardware becomes a quadrotor for 60~s hops to new vantage points. Mode changes occur in under 6~s via indexed hinges and Hall-effect latches triggered from a tablet interface, letting an operator reposition the robot when rubble or stairs block the rolling path.

\begin{figure}[h]
    \centering
    \includegraphics[width=0.85\linewidth]{images/m4img-001.png}
    \caption{M4-inspired rotor-wheel transformations guiding the Mini Morphobot's two-mode concept.}
    \label{fig:modes}
\end{figure}

\subsection{Conceptual Interest and Motivation}
The most compelling idea is appendage repurposing: a single electromechanical module provides traction, thrust, and stabilization, echoing M4's research insight while staying achievable with student resources. Instead of spreading effort across many locomotion styles, we concentrate on robust drive-and-fly transitions that matter to first responders. Success would demonstrate that a low-mass, battery-operated scout can reposition in 3D space without a separate drone, shrinking the kit responders carry and simplifying operator training.

\subsection{Functional Components}
\begin{enumerate}[label=\textbf{FC\arabic*}:, leftmargin=1.5cm]
    \item \textbf{Morphing Airframe and Locks} --- Carbon sandwich chassis with indexed hinge blocks, spring-loaded detents, and wiring passthroughs that hold each pod rigid in both ground and flight attitudes.
    \item \textbf{Convertible Rotor-Wheel Pods} --- Custom ducts with bonded tires, 3D-printed hub carriers, and a shoulder gearbox that repositions the pods with $<1^\circ$ repeatability.
    \item \textbf{Ground Drivetrain} --- In-wheel brushless motors with planetary reductions and field-oriented control for 0.8~m/s cruise plus differential steering.
    \item \textbf{Flight Powertrain} --- 5-inch propellers, 1200~kV outrunners, and BLHeli ESCs sized for 1.5~kg thrust allowing stable quadrotor hops carrying the entire chassis.
    \item \textbf{Perception and Control Stack} --- Teensy 4.1 motor controller linked to a Pixhawk or Cube Orange flight controller running PX4, with ROS~2 teleop on a Jetson Nano for shared state estimation.
    \item \textbf{Power and Communications Backbone} --- 4S Li-ion battery with dual buck converters, current sensing, and Wi-Fi + RF telemetry for command and data logging.
\end{enumerate}

\subsection{Mermaid System Breakdown}
\begin{figure}[h]
\centering
\begin{lstlisting}
```mermaid
flowchart TD
    Start["Operator Command"]
    Start --> Mode{Select Mode}
    Mode -->|Drive| GroundFC[Ground Drivetrain]
    Mode -->|Fly| FlightFC[Flight Powertrain]
    GroundFC --> Locks[Morphing Locks]
    FlightFC --> Locks
    Locks --> Sensors[Perception + Control Stack]
    Sensors --> Power[Power + Comms Backbone]
    Power --> Start
```
\end{lstlisting}
\caption{Mermaid flowchart summarizing subsystems needed for drive/fly operations.}
\label{fig:mermaid}
\end{figure}

\section{Concept Questions}
\begin{enumerate}[label=\textbf{Q\arabic*}:, leftmargin=0pt, itemsep=1.2em]
    \item \textbf{What is the value of your product to the end-user?}\\
    Urban search and rescue teams gain a single robot that can roll quietly through doorways and then quickly hop over collapsed stair segments to reach trapped occupants. The mission storyboard in Fig.~\ref{fig:sar} illustrates how the operator stays at street level, teleoperates wheeled inspection, and only flies when debris gaps appear, minimizing battery use yet preserving access flexibility. This blend shortens mission timelines and reduces the number of specialized robots responders must stage at the incident.
    \item \textbf{What is the closest alternative to your product?}\\
    The closest alternative is deploying a small quadrotor alongside a separate rugged UGV (e.g., DJI Mavic + FLIR FirstLook). While that pair delivers comparable capabilities, it doubles logistics (two batteries, controllers, pilots) and forces responders to recover and redeploy hardware between modalities. Our Mini Morphobot trades heavy payload capacity for convenience: the same rotor-wheel pods stay on the robot, so mode changes are software triggers instead of equipment swaps.
    \item \textbf{What is the metric of success for your product? How will you measure it?}\\
    We will declare success if the robot (i) completes a ground-to-flight-to-ground cycle in under 12~s, (ii) sustains 0.8~m/s average speed in drive mode for 15~m, and (iii) performs a 5~m horizontal flight hop carrying its full chassis. Timing gates, encoder logs, and motion capture at the sandbox will provide objective data for each criterion.
    \item \textbf{Pick one aspect to polish to a finished product.}\\
    The mechanical morphing mechanism and locks will reach a polished standard: smooth bearing-supported hinges, hidden wiring, magnetic detents that audibly click, and covers that protect against dust during takeoff. A rough prototype (visible wobble, exposed wires, detents that need hand assistance) is explicitly unacceptable and will be engineered out through iterative machining and cable routing.
    \item \textbf{Which aspect(s) will you not develop to a finished product?}\\
    Full autonomy and onboard perception remain experimental. We will run ROS~2 teleoperation and simple PX4 position hold, but we will not harden SLAM, obstacle avoidance, or AI-based victim detection. Those features require datasets and validation far beyond the term, so we treat them as scaffolding for operator-driven demos.
    \item \textbf{What is your most critical module and why?}\\
    The convertible rotor-wheel pod (FC2) is most critical because it must satisfy conflicting requirements: structural stiffness for fast driving, low mass for flight, and precise alignment for vibration-free hover. Failure to meet any requirement instantly degrades both locomotion modes, so we will prototype this module first, run torsion/balance tests, and only then replicate it across all four corners.
    \item \textbf{What kind of data infrastructure will you need? How will you test it?}\\
    Our ROS~2 infrastructure will stream wheel encoder ticks, IMU data, hinge position, current draw, and PX4 state vectors over Wi-Fi to a logging laptop. A PostgreSQL-backed rosbag2 pipeline will archive every trial. We will inject mock data using Gazebo/PX4 SITL to ensure the telemetry topics saturate at expected bandwidth, then run hardware-in-the-loop bench tests where the Pixhawk publishes synthetic flight data while the Teensy reports wheel motion, verifying synchronization before field trials.
\end{enumerate}

\begin{figure}[h]
    \centering
    \includegraphics[width=0.9\linewidth]{images/m4img-002.png}
    \caption{Caltech's SAR storyboard conveying the value proposition for responder end-users.}
    \label{fig:sar}
\end{figure}

\begin{figure}[h]
    \centering
    \includegraphics[width=0.9\linewidth]{images/m4img-003.png}
    \caption{Reference workflow for morphing appendages, guiding our subsystem architecture.}
    \label{fig:workflow}
\end{figure}

\end{document}
