\documentclass[11pt]{article}
\usepackage[margin=1in]{geometry}
\usepackage{enumitem}
\usepackage{microtype}
\usepackage{graphicx}
\usepackage{listings}
\usepackage{amsmath}
\usepackage{amssymb}
\usepackage{cleveref}
\usepackage{float}
\usepackage{booktabs}
\usepackage{parskip}




\lstset{basicstyle=\ttfamily\small, frame=single}

\title{Concept Proposal: Transformer Robot}
\author{Gyan Edbert Zesiro (ID:38600060) \and Ryan Edric Nashota (ID:33508219)}
\date{\today}

\begin{document}
\maketitle

\section{Concept Overview}
\subsection{Goal and Vision}
We will build a simpler version of Caltech's Multi-Modal Mobility Morphobot (M4)\footnote{E.~Sabree et~al., 
``Multi-Modal Mobility Morphobot (M4) with appendage repurposing for locomotion plasticity enhancement,'' \textit{Nature Communications}, 2023.}. 
The original M4 platform can transform between six locomotion modes, however due to the
time and resource constraints of this project, we will narrows the scope to two 
high-impact modes: 
\begin{itemize}
    \item \textbf{Ground Mode:} Agile wheeled driving indoors.
    \item \textbf{Flight Mode:} Short-hop quadrotor flight outdoors.
\end{itemize}
Our goal is to design and build a morphing robot that can switch between these two modes, and most importantly, learn from the challenges up ahead. We do want to learn as much as possible, so we will implement the following component from scratch:
\begin{itemize}
    \item Mechanical Design
    \item Flight Controller
    \item Control System
    \item Power System 
\end{itemize}
We recognize the complexity of designing these components, for example building flight controller alone would require deep understanding of control theory, sensor fusion, and embedded systems. However, this is our final year and we would like to end it with a good note.

The vision and purpose of the device is that by utilizing the same set of rotor-wheel appendages,
we can create a more versatile locomotion device, allowing it to be used in multiple scenarios. 
For example, the robot could drive efficiently through a building's corridors to survey damage, 
then switch to flight mode to cross a collapsed stairwell or navigate between floors through 
an open atrium, enabling a single device to navigate complex environments.

Sometimes it will be hard for flying drone to cross tight spaces, as it might cause damage and ultimately break the robot.
So in such cases it would be beneficial for the drone to change into wheeled mode to pass through narrow passages safely.

\begin{figure}[H]
    \centering
    \includegraphics[width=0.5\linewidth]{images/m4_caltech.png}
    \caption{M4 Robot from Caltech.}
    \label{fig:modes}
\end{figure}

\subsection{Form Factor and Operating Concept of Final Product}
The robot looks like a remote-control car with four motorized wheels arranged in an 
"X" pattern on a frame as seen in \Cref{fig:sketch_simple}. What makes it special is 
that each wheel is actually part of a convertible rotor-wheel pod, where it is 
essentially a propeller enclosed in a protective duct with a tire attached to its outer rim.

When driving on the ground, the ducts lock flat at $0^\circ$ and the 
motors spin the tires like a normal RC car, complete with active suspension for smooth 
movement over uneven terrain. When the robot needs to fly, say, to get over obstacles or 
climb stairs, the four pods tilt upward to a $90^\circ$ angle, the propellers spin up, 
and the whole system transforms into a quadcopter drone. Since the robot might be heavy,
our goal is to give the robot enough flight time to hop over barriers and reach new locations, not a 
high altitude flight found in most drones.

Switching between driving and flying modes should be quick, currently we are targeting it to be less than
10 seconds. The transformation happens through motorized hinges that click into preset positions, 
with sensors confirming for feedback to ensure they are in the correct position. 
An operator controls everything from a RC controller, deciding when to drive and 
when to take flight based on what obstacles lie ahead.

\begin{figure}[H]
    \centering
    \includegraphics[width=0.7\linewidth]{images/2modes.jpeg}
    \caption{Sketch of Transforming Robot in (left) Ground Mode and (right) Flight Mode.}
    \label{fig:sketch_simple}
\end{figure}


\subsection{Conceptual Interest and Motivation}
The most compelling idea is appendage repurposing, similar to how animals and humans use
their limbs for multiple functions. We have to design a single electromechanical 
module which provides traction, thrust, and stabilization, 
echoing M4's research insight while staying achievable with the given time and resources in MECH421/423. 

\begin{figure}[H]
    \centering
    \includegraphics[width=0.7\linewidth]{images/mechanism.jpeg}
    \caption{Preliminary Concept Sketch of Transforming Mechanism}
    \label{fig:sketch_transform}
\end{figure}

Personally, both of us wanted to do further study in navigation of mobile robots, and as such
this project provides another avenue for our graduate studies. Unlike typical drones or rovers,
which navigation algorithm have been well studied, a morphing robot presents new challenges
in state estimation, control, and path planning. We think that this project will be good to 
expose ourself on hese topics which we want to research in the future. 

Finally, we think it's just cool. As Richard Feynman once said, ``Physics is like sex: sure, 
it may give some practical results, but that's not why we do it.'' The same applies here, while 
this robot has practical applications in search-and-rescue, the real motivation is the challenge 
and the learning experience. Even if the final prototype doesn't achieve full autonomous 
flight-to-ground transitions, the process of designing the transformation mechanism, creating a working PCB, integrating the control systems, and wrestling with the state estimation problems will teach us 
far more than building yet another standard quadcopter or RC car. We'd rather aim high and 
learn from the inevitable obstacles than play it safe with a project we could finish in our sleep.
Afterall a wrong answer isn’t a meaningless one.


\subsection{Functional Components}

Table~\ref{tab:functional_components} summarizes the six main subsystems that enable the robot's dual-mode operation.

\begin{table}[h]
\centering
\caption{Functional Components of the Morphing Robot}
\label{tab:functional_components}
\begin{tabular}{p{0.3\textwidth} p{0.55\textwidth} p{0.1\textwidth}}
\toprule
\textbf{Component} & \textbf{Description} & Responsibility\\
\midrule
\textbf{FC1: Morphing Airframe} & Lightweight PLA chassis with motorized indexed hinges and Hall-effect sensors to lock pods at 0° (ground) and 45° (flight) positions. Target transition time under 10 seconds. If budget permits we will use fancier material such as carbon fiber to reduce weight and improve durability.& Ryan \\
\addlinespace
\textbf{FC2: Rotor-Wheel Pods} & Four identical convertible modules, each containing a ducted rotor with tire bonded to the rim. Provides both ground traction and aerial thrust using the same hardware. & Ryan \\
\addlinespace
\textbf{FC3: Ground Drivetrain} & Brushless hub motors with gearboxes drive the tires for wheeled locomotion. Differential steering enables agile indoor navigation. & Gyan\\
\addlinespace
\textbf{FC4: Flight Powertrain} & Same motors repurposed as quadcopter propulsion with appropriate propellers and ESCs, enabling short flight hops to clear obstacles. & Gyan\\
\addlinespace
\textbf{FC5: Control System} & RC transmitter for manual mode switching and teleoperation. Flight controller (e.g., Pixhawk with PX4 firmware) for stabilization. Optional onboard computer for future autonomous navigation research. & Gyan\\
\addlinespace
\textbf{FC6: Power System} & Lithium-polymer battery with voltage regulation for motor and electronics power. Sized to support extended ground operation and multiple flight cycles per charge. & Ryan\\
\bottomrule
\end{tabular}
\end{table}

\begin{figure}[h]
    \centering
    \includegraphics[width=0.6\linewidth]{images/flow_chart.png}
    \caption{Flow Chart of Subsystem for Transforming Robot}
    \label{fig:flow_chart}
\end{figure}


\section{Concept Questions}
\begin{enumerate}[label=\textbf{Q\arabic*}:, leftmargin=0pt, itemsep=1.2em]
    \item \textbf{What is the value of your product to the end-user?}\\
    The added flexibility from the multiple modes allow a single robot to handle multiple scenarios.
    For example, urban search and rescue teams gain a single robot that can roll quietly 
    through doorways and then quickly hop over collapsed stair segments to reach 
    trapped occupants. The mission storyboard in \Cref{fig:sar} 
    illustrates how the operator stays at street level, teleoperates wheeled inspection, 
    and only flies when debris gaps appear, minimizing battery use yet preserving access 
    flexibility. This blend shortens mission timelines and reduces the number of specialized 
    robots responders must stage at the incident.

    \begin{figure}[H]
    \centering
    \includegraphics[width=0.9\linewidth]{images/use_case.png}
    \caption{Caltech's SAR storyboard conveying the value proposition for responder end-users.}
    \label{fig:sar}
    \end{figure}

    We would imagine that advancement of this technology will also find applications in
    other fields such as space exploration (e,g, Mars rover), agriculture (e.g., inspecting crops on the ground and flying over
    obstacles), military reconnaissance (e.g., driving through urban environments and flying
    over barriers), and environmental monitoring (e.g., ground surveys and aerial mapping).
    
    For our demonstration purposes though, we will show that the robot can navigate through tight spaces that a usual drone cannot pass through, by switching to wheeled mode.


    \item \textbf{What is the closest alternative to your product?}\\
    Since the project is inspired by Caltech's M4, the research done on M4 is the best alternative to our product. 
    But if we are only concerend with driving and flying robots, there are other academic prototype such as Caltech's ATMO (Aerially Transforming Morphobot), which also uses four 
    thrusters that double as wheels and can transform between quadcopter and ground rover modes 
    using a single motor mechanism.  

    Another recent example is Virginia Tech's bistable morphing robot, which uses shape memory 
    alloy actuators to rapidly switch between flying and driving configurations in under 50 milliseconds.  
    Both of these platforms focus specifically on the flight-to-ground transition problem we're addressing.
    
    Outside of research prototypes, there isn't really a direct commercial equivalent. The practical alternative is simply using \textbf{two separate robots}: a standard quadcopter drone for aerial reconnaissance and an RC car or small ground rover for close-up inspection. This is what most hobbyists and even some professional applications currently do, switching between devices based on the terrain.

    \item \textbf{What is the metric of success for your product? How will you measure it?}\\
    We will declare success if the robot (i) completes a ground-to-flight-to-ground cycle in 
    under 20~s, (ii) sustains 0.8~m/s average speed in drive mode for 15~m, and 
    (iii) performs a 5~m horizontal flight hop carrying its full chassis. 
    We will use timing gates, encoder logs, and motion capture to quantify these metrics.
    
    \item \textbf{Pick one aspect to polish to a finished product.}\\
    The mechanical morphing mechanism and locks will reach a polished standard: 
    smooth bearing-supported hinges, hidden wiring, 
    magnetic detents that audibly click, and covers that protect against dust during takeoff. 
    A rough prototype (visible wobble, exposed wires, detents that need hand assistance) is explicitly 
    unacceptable and will be engineered out through iterative machining and cable routing.
    \item \textbf{Which aspect(s) will you not develop to a finished product?}\\
    Full autonomy will be hard to achieve, but we will still do the work, to the level where the robot can't crash easily.
    We will run ROS~2 teleoperation and simple PX4 position hold, 
    but we will not harden SLAM, obstacle avoidance, or object detection (most movement will be done with manual operation), we don't think this is necessary for this project, and is currently a nice to have feature. 
    Additionally, features which require datasets and validation (e.g. machine learning model) will not be implemented because of time constraints, and our lack of knowledge in running a machine learning model in tiny embedded systems.
    
    \item \textbf{What is your most critical module and why?}\\
    The convertible rotor-wheel pod (FC2) is most critical because it must 
    satisfy conflicting requirements: structural stiffness for fast driving, 
    low mass for flight, and precise alignment for vibration-free hover. 
    Failure to meet any requirement instantly degrades both locomotion modes, so we will 
    prototype this module first, run torsion/balance tests, and only then replicate it across all 
    four corners.
    \item \textbf{What kind of data infrastructure will you need? How will you test it?}\\
    Our ROS~2 infrastructure will stream wheel encoder ticks, IMU data, 
    hinge position, current draw, and PX4 state vectors over Wi-Fi to a logging laptop. 
    A PostgreSQL-backed rosbag2 pipeline will archive every trial. 
    We will inject mock data using Gazebo/PX4 SITL to ensure the telemetry topics 
    saturate at expected bandwidth, then run hardware-in-the-loop bench tests where the 
    Pixhawk publishes synthetic flight data while the Teensy reports wheel motion, 
    verifying synchronization before field trials.
\end{enumerate}

\end{document}
